\chapter{Introduction to EiC}

{\bf E}xtensible {\bf i}nteractive {\bf C}, EiC, is a hand crafted,
recursive--descent C interpreter. To start EiC simply enter at the
system prompt (\T{\%}):
\begin{quote}
\begin{verbatim}
% eic
\end{verbatim}
\end{quote}

As the name implies, EiC is interactive, but see section
\SecPage{NonInteractive} for running EiC in batch mode.  Basically the user
enters commands, or {\it immediate statements} at the interpreter
prompt {\tt EiC \#>}, where the hash mark represents the current line
number; for example:
\small
\begin{quote}
\begin{verbatim}
EiC 1> #define PI  3.14159
EiC 2> PI * strlen("Hello, World!\n");
\end{verbatim}
\end{quote}
\normalsize

and then EiC will respond with:
\small
\begin{quote}
\begin{verbatim}
43.98226
\end{verbatim}
\end{quote}
\normalsize

where {\tt strlen}, see \SecPage{stringh}, is a standard C function
that will return the number of characters in the argument string
\verb+"Hello, World!\n"+:
\small
\begin{quote}
\begin{verbatim}
EiC 3> strlen("Hello, World!\n");
        14
\end{verbatim}
\end{quote}
\normalsize
In fact, virtually all of the C runtime library and a good proportion
of the POSIX.1 libray are supported by EiC (but see \ChpPage{Library},
for details).

To exit EiC simply enter:

\small
\begin{quote}
\begin{verbatim}
EiC 4> :exit
\end{verbatim}
\end{quote}
\normalsize

EiC is a bytecode~\index{bytecode}\cite{budd87} compiler
that generates its own internal intermediate language known as stack code,
\index{stack code} which is a bit like the Pascal P-code system
\cite{pemberton-daniels82}. It executes the stack code via its own internal 
stack machine\index{stack machine}.  The intermediate code produced
from the previous call to {\tt strlen} is:

\small
\begin{quote}
\begin{verbatim}
0:pushptr  245980           4:stoval                            
1:bump     1                5:pushint  1          
2:checkar  1 1              6:call                
3:pushptr  398192           7:halt                
\end{verbatim}
\end{quote}
\normalsize


While the details of the stack code will not be discussed in this
document its usage means that executed commands generally perform much
faster than an interpreter that uses no intermediate code.

\section{\label{sec:EiCvsC} EiC vs C}

Because EiC is interactive it differs from C in several
ways. In this section I will outline what is currently
missing from EiC and how EiC differs from ISO C. 

Although EiC can parse almost all of the  C programming
language~\cite{kernighan-ritchie88} right up front it is
best to mention what is currently lacking or different:
\begin{enumerate}

\item EiC is pointer safe\index{pointer!safe}. It detects many classes
of memory read and write violations
(see~\SecPage{ArrayBoundChecking}). To help in interfacing
compiled library code to EiC, EiC uses the pointer-qualifiers
\T{safe} and \T{unsafe}, see \SecPage{PointerPragmas}.


\item Structure \index{structure! bit fields}\index{bit fields} bit fields 
 are not supported.

\item 

While structures and unions can be returned from and passed by value to
functions it is illegal in EiC to pass a structure
or a union to a variadic function (that is, a function that
takes a variable number of arguments):
        \begin{quote}
        \begin{verbatim}
EiC 1> struct stag {int x; double y[5];} ss; 
EiC 2> void foo(const char *fmt, ...);  
EiC 3> foo("",ss);
Error: passing a struct/union to variadic function `foo'
        \end{verbatim}
        \end{quote}

\item The C concept of linkage\index{linkage!external} is not supported.
        This is because EiC does not export identifiers to a linker --
        as does a true C compiler. EiC works from the concept of a
        single {\it translation unit}, see
        \SecPage{translationUnit}. However, static global variables
        remain private to the file they are declared in, see
        \Page{item:static}


\item EiC does not parse preprocessor numbers\index{pp numbers}, 
        which aren't valid numeric constants; for example, {\tt
155.6.8}, which is an extended floating point constants will cause an
error.

\item EiC supports both standard C like comments \verb+/* ... */+
        and C++ style comments (see section~\SecPage{Comments}). Also, when
	EiC is run in script mode (see \SecPage{ScriptMode}) it
	treats all lines that start with `\T{\#}' and which can't be
	interpreted as a preprocessor directive as a comment.

\item  There are no default type specifiers for function return values. In EiC it is 
        illegal to not explicitly state the return type of a function:

\begin{quote}
\begin{verbatim}
foo() { ... }     /*   error: missing return type */
int foo() { ... } /* correct, return type specified */ 
\end{verbatim}
\end{quote}

\item  In addition to function definitions and declarations with 
an empty parameter list EiC only supports prototype declarations and
definitions:

\begin{quote}
\begin{verbatim}
int foo(); /* Empty parameter list allowed */ 
int f(value) int value { ... } /* Illegal: old style C */ 
int f(int); /* Allowed, prototype declaration */ 
int f(int value); /*Allowed, full prototype declaration */
\end{verbatim}
\end{quote}

\item EiC does not support trigraph sequences, wide characters or  
        wide strings: nor does it support the standard header
        \verb+<locale.h>+.

\item EiC's preprocessor lacks the \directive{line} directive.

\item For convenience, EiC allows the \directive{include} directive
        to have an extra form that permits the parsing of a {\it
        token-sequence} in the form \verb+#include filename+; that is,
        without enclosing double quotes or angled brackets (see
        section \SecPage{Include}).

\item Besides parsing preprocessor directives (\ChpPage{Preprocessor})
or C statements (\ChpPage{Specifications}), EiC also parses its own
internal house keeping language (see section \SecPage{EiCcommands}).
House keeping commands are communicated to EiC via  lines that
begin with a colon.

\end{enumerate}

\section{Running EiC}
\label{sec:RunningEiC}

To run EiC interactively just enter \T{eic} at your system
prompt:
\begin{quote}
\begin{verbatim}
% eic
\end{verbatim}
\end{quote}

However, you also need to set the environmental variable
\T{HOMEofEiC}\index{HOMEofEiC}, this is so that EiC knows where to find
its include files etc.  The
\T{HOMEofEiC} environmental variable must be set to point to the directory
that contains the EiC include directory. For example: \T{\$HOME/EiC} or
\T{/usr/local/EiC}.

In bash, ksh or zsh use:
\begin{quote}
\begin{verbatim}
% export HOMEofEiC=...
\end{verbatim}
\end{quote}

In tcsh enter:
\begin{quote}
\begin{verbatim}
% setenv HOMEofEiC ...
\end{verbatim}
\end{quote}

where the dots represent the name of the EiC directory

You may wish to include the command in one of your startup scripts
such as the \T{.cshrc} or the \T{.bashrc} file.

\subsection{EiC immediate instructions}
\label{sec:ImmediateStatements}
\index{immediate statement}
\index{statement!immediate}

In interactive mode the user interacts directly with the EiC
interpreter.  He or She enters C statements, C declarations,
preprocessor directives or EiC interpreter commands at
EiC's command line prompt:

\begin{quote}
\begin{verbatim}
EiC 1> 
\end{verbatim}
\end{quote}

The number before the closing angled bracket represents the current
line number.  

As a user types out an instruction EiC is analysing the input
character stream and is checking for matching brackets. When a
closing bracket, either square or curved, is entered, EiC moves the
cursor quickly to its matching opening bracket and back again. This
can be especially helpful when entering commands with complicated and
nested bracketing. However, EiC will produce a beep if an opening
bracket cannot be matched to the current closing bracket.

Each immediate instruction produces a type, even if the type is 
\T{void};\index{void} as for example, C statements, declarations etc. 
All resulting types and their values are displayed:


\begin{quote}
\begin{verbatim}
EiC 1> 3*55.5;
        166.5
EiC 2> "hello, world!";
        hello, world!
EiC 3> int i;
        (void)
EiC 4> for(i=0;i<10;i++);
        (void)
EiC 5> i;
        10
EiC 6> struct {int a, b;} ab = { 5,3};
        (void)
EiC 7> ab;
        {5,3}
EiC 8> ab.a = 3;
        3
\end{verbatim}
\end{quote}

While arrays are not expanded, the EiC interpreter display routine
considers all \T{char} pointers to be valid null-character terminated
strings (see section~\SecPage{StringConstants}) and will display them
as a such. However, as a safe guard against runaway sequences it 
restricts such print outs to at most 100 characters. 


\subsection{EiC error recovery}
\index{error recovery}
\label{sec:ErrorRecovery}

It is possible to interrupt\index{interrupt!immediate instruction} the
execution of any immediate instruction by pressing control C:

\begin{quote}
\begin{verbatim}
EiC 9> while(1);  /* loop forever or until interrupted by <Ctl>C */ 
EiC interrupted file ::EiC::, line 9
EiC: error clean up entry pt 0
EiC 10> 
\end{verbatim}
\end{quote}

The information displayed between line 9 and 10 is informing that EiC
was interrupted at line 9 in file \T{::EiC::}, which is the name of
the interpreter's command line. The next line of information informs
that EiC has entered automatic error recovery and garbage
collection\index{garbage collection}, or clean up at entry pt 0. The
various entry points are of no real concern. Instead, what is
interesting is that all new memory allocations, data types, macro
definition etc, will be cleaned up and removed. It is EiC's way of
trying to set its interpreter back to the state it was before line 9
was entered. In this case there is nothing to clean up but if on
line 9 I had entered say \verb+#include foobar.c+, which contained
errors, then there could have been potentially thousands of pieces of
information to clean up.

While EiC's clean up operation attempts to regain a previous state,
it is not always 100\% successful; for example, say I have already
included the file \T{foobar.c} then I decide to make some changes to
it and then re-include it, but I inadvertently enter a typo, an
error, somewhere in the file \T{foobar.c}. EiC will detect the error
and when it is finished translating the entire unit it will trigger
the clean up procedure. The clean up operation will then remove all
traces of the contents of \T{foobar.c} from the interpreter and the
resulting state will be as if it had never been entered.


\subsection{Entering multi line commands}
\label{sec:EditMultipleLines}

EiC has a command line \index{editor!editing commands} editor where
the user can use the delete key and the left or right arrow key to aid
editing or one or more of the following commands:

\begin{quote}
\begin{verbatim}
printable characters print as themselves (insert not overwrite)
^A moves to the beginning of the line
^B moves back a single character
^E moves to the end of the line
^F moves forward a single character
^K kills from current position to the end of line
^P moves back through history
^N moves forward through history
^H and DEL delete the previous character
^D deletes the current character, or EOF if line is empty
^L/^R redraw line in case it gets trashed
^U kills the entire line
^W kills last word
<LF> and <CR> return the entire line regardless of the cursor position
\end{verbatim}
\end{quote}

\verb+^A+ indicates that the control key is held down while simultaneously
pressing the A key, in either upper or lower case.


To input commands into EiC that require multiple lines of code you
can just add the backslash `\verb \ ' character at the end of each
line to be continued. For example:

\begin{quote}
\begin{verbatim}
EiC 1> double sqr(double x) \
EiC 2> {\
EiC 3>     return x*x;\
EiC 4> } 
\end{verbatim}  
\end{quote}


However, as EiC does not supply a full screen editor,\index{editor}
a second method for entering multi line commands is to use EiC's C
preprocessor to include a sequence of external declarations via file
inclusion. For example:

\begin{quote}
\begin{verbatim}
EiC 1> #include "examples/sqr.c"
\end{verbatim}
\end{quote}

and where the contents of {\tt examples/sqr.c} is:

\begin{quote}
\begin{verbatim}
#include <math.h>
double sqr(double x) 
{
   return x*x;
} 
\end{verbatim}  
\end{quote}


\subsection{EiC on start up}
\label{sec:starteich}
\index{EiC!starteic}

When EiC is launched it automatically looks for a
\T{starteic.h} file. EiC first looks for this file in the current
working directory then if this fails, it looks in your home directory
and finally if all else fails, it looks in EiC's system
\T{include} directory. It is not a big deal if EiC cannot find a
\T{starteic} file, however the purpose for this file is to allow you to 
specify defaults on EiC start up.

The default system \T{starteic.h}, stored in
EiC's include directory should resemble:

\begin{quote}
\begin{verbatim}
#ifndef _STARTEiCH
#define _STARTEiCH

/* ISO STUFF */
#include <stdio.h>
#include <stdlib.h>
#include <ctype.h>
#include <stdarg.h>
#include <string.h>
#include <math.h>
#include <float.h>
#include <limits.h>
#include <errno.h>
#include <assert.h>

/* POSIX.1 STUFF */
#include <fcntl.h>

#endif /* _STARTEiCH */
\end{verbatim}
\end{quote}

The start up procedure allows you to define local defaults within a
given directory, your own global defaults within your home directory,
and finally the EiC system defaults. A possible home directory
\T{starteic.h} might look something like:

\begin{quote}
\begin{verbatim}
/* include system headers */
#include <starteic.h>

/* standing macros */
#define sys(x) system(#x)
#define pwd  system("pwd");
#define ls   system("ls");
#define help(x)    system("man 3 " #x)
\end{verbatim}
\end{quote}

Note, that lines beginning with {\tt `:'} are EiC command directives,
which are explained in detail in section \SecPage{EiCcommands} and
that the home directory \T{starteic.h} file includes
EiC's system \T{starteic.h}. Therefore, it is assumed
that any local \T{starteic.h} file would inturn include this
file. 

This facility can also be switched off by passing EiC the
\T{-N}\index{command line switch!-N} command line switch on start up,
see \SecPage{EiCCommandLineOptions}.


\subsection{EiC command line switches}
\label{sec:EiCCommandLineOptions}
\index{EiC! command line options}

You can also change the default behaviour of EiC on start up, by using
one or more of EiC's command line swtiches:

\begin{quote}
\begin{verbatim}
                EiC
 An Extensible Interactive C interpreter
 To start eic, type eic.
 To exit eic, type :exit.

 Usage:
        eic  [-Ipath] [-Dname[=var]] -[hHvVcCrR]  [[file] [fileargs]]
 Options:
   C preprocessor directives:
        -Ipath      search for include files in path
        -Dname      define a symbolic name to the value 1
        -Dname=var  define a symbolic name to the value var
                    Note, there is no spaces allowed
   EiC directives:
        -h -H       causes this usage to be displayed
        -v -V       print EiC's Log information
        -p          showline
        -P          show path of include files
        -t -T       turns trace on
        -c -C       turns timer on
        -e          echo HTML mode
        -r          restart EiC. Causes EiC to be re initiated
                       from the contents of EiChist.lst file
        -R          same as `r', but prompts the user to accept
                       or reject each input line first
        -s -S       run silently
        -f          run in script mode
        -n          no history file
        -N          don't use any startup.h files
        -A          Non-interactive-mode
        file        EiC will execute `file' and then stop; for example:
                      % eic foo.c 
        fileargs    command line arguments, which get passed onto file
\end{verbatim}
\end{quote}

This above listing duplicates the response of EiC to:
\begin{quote}
\begin{verbatim}
% eic -h
\end{verbatim}
\end{quote}\index{command line switch!-h}

The \T{showline} option is discussed on page~\pageref{item:showline};
the \T{trace} option on page~\pageref{item:trace}; the preprocessor
directives on page~\pageref{item:comm-switch}; the \T{-P} includes
option on page~\pageref{item:includes}; the \T{-c} option on
page~\pageref{item:timer}; the \T{-v} option on
page~\pageref{item:verbose}; and the \T{-N} option is discussed in
section
\SecPage{starteich}; while the \T{-r}, \T{-R},
\T{-n}, \T{-A}, \T{-f}, \T{file} and
\T{fileargs} options are discussed below.

\subsection{EiC history file}
\label{sec:Historyfile}
\index{EiC!history file}
\index{history file}

During an EiC interactive session each command line entered that does
not cause an error is saved in a history file
\T{EiChist.lst}\index{EiChist.lst}\index{history file!EiChist.lst} in
the directory that EiC was launched from; that is, each unique
directory used to launch EiC will have its own \T{EiChist.lst}
file. The file is normally created new on each start up and the
previous contents (if existing) are ignored. That is, unless the command line
switch \T{-r}\index{command line switch!-r} is used:

\begin{quote}
\begin{verbatim}
% eic -r
\end{verbatim}
\end{quote}

The switch \T{-r} used on EiC start up informs EiC to enter its
re-initialization mode and the commands stored in the file
\T{EiChist.lst} will be re executed in order of occurrence. 
The contents of the file
\T{EiChist.lst} is then retained and used to from the start of the
history list (discussed on page~\pageref{item:history}) of the new
session. The main purpose for this file is to provided the user with
an automatic method for recording an EiC session and provide a way to
quickly recapture a previous EiC session. Thus, allowing the user to
resume from where he or she left off.


When re-initailizing EiC, it is not always desirable to execute every
instruction line in the \T{EiChist.lst} file. You may wish to edit it
first to remove or modify certain lines. To aid in this process it is
also possible to re-initialize EiC via the uppercase
\T{R}\index{command line switch!-R} switch. In this case the user is
offered the opportunity to either input the current line, \T{Y}, to
edit the current line before input, \T{E}, or to not include the
current line \T{N}; for example (where the contents of \T{EiChist.lst}
is \T{int a, b, c;}):

\begin{quote}
\begin{verbatim}
% eic -R
...
Re Initiating EiC -- please wait.
Re-enter [int a, b, c;] (Y/N/E)?     
\end{verbatim}
\end{quote}


The history file mechanism can also be switched off by selecting
the \T{-n}\index{command line switch!-n} command line option on start up:
\begin{quote}
\begin{verbatim}
% eic -n
\end{verbatim}
\end{quote}

While the following allows EiC to be re-initialized, it prevents
EiC from creating a new \T{EiChist.lst} file: 

\begin{quote}
\begin{verbatim}
% eic -rn
\end{verbatim}
\end{quote}

The old file \T{EiChist.lst} file will be retained and used and no further
commands will be added to the list:



\subsection{EiC non-interactive mode}
\label{sec:NonInteractive}
\index{EiC!non-interactive mode}

There are two modes for running EiC and while this document is
primarily concerned with EiC's interactive mode, EiC can also be run
non-interactively. For example, the following is used to execute the program
\T{examples/hello1.c} in EiC's \T{examples} directory:

\begin{quote} 
\begin{verbatim}
% eic examples/hello1.c
\end{verbatim}
\end{quote}

The above command uses the \T{file} option that instructs EiC to
load the file \T{hello1.c}, compile it into bytecode, execute it and
then to stop.  The program, \T{hello1.c}, is assumed to be a self
contained C program, which contains the definition of a {\tt
main}\index{main} function. The \T{main} function is used to establish
the start point of the program:

\begin{quote} 
\begin{verbatim}
#include <stdio.h>
void message(void)
{
    const char *s = "Hello, world!";
    puts(s);
}

int main(void)
{
    message();
    return 0;
}
\end{verbatim}
\end{quote}

The entire file, {\tt hello1.c}, plus all it includes is considered
to be a single translation unit, see
section~\SecPage{translationUnit}. Also, the default procedure for
including \T{starteic.h} files is ignored and no \T{EiChist.lst} is
utilized. The options for modifying EiC's non-interactive behaviour is
limited to the command line options specified in
section~\SecPage{EiCCommandLineOptions}.

It is also possible to write programs that take command line arguments
in the usual C way, as seen from \T{examples/main2.c}:

\begin{quote} 
\begin{verbatim}
#include <stdio.h>
int main(int argc, char **argv)
{
    while(argc--)
        printf("%s\n",*argv++);
    return 0;
}
\end{verbatim}
\end{quote}

The first parameter, which is normally called \T{argc}, holds the
number of argument strings passed to the program and is always at
least one. The second parameter, which is normally called
\T{argv} is an array of unspecified size of pointers to the input  strings, 
which the first one will be the name of the program being executed:

\begin{quote} 
\begin{verbatim}
% eic examples/main2.c 123 hello -Dworld this.and.that
examples/main2.c
123
hello
-Dworld
this.and.that
\end{verbatim}
\end{quote}


\subsubsection{Running EiC in script mode}
\label{sec:ScriptMode}
\index{EiC!script mode}
\index{scripts!EiC}

In non-interactive mode EiC runs generally like a typical
interpreter, accepting input from a complete C program. However, EiC
can also run shell scripts non-interactively. For the following
examples, in this section only, it will be assumed that you are in
EiC's directory \T{../EiC/module/examples} and that
\T{eic} is installed in \T{/usr/local/bin}.

Below is an example of an EiC script, called \T{hello.eic}:
\begin{quote} 
\begin{verbatim}
#!/usr/local/bin/eic -f

#include <stdio.h>
printf(" ******* Hello from EiC's script mode. ******\n");
\end{verbatim}
\end{quote}
The \T{-f}\index{command line switch!-f} command-line switch, informs
EiC to run in script mode.  In script mode, EiC will treat all lines
beginning with \T{\#} and which cannot be interpreted as a
preprocessor directive (see~\ChpPage{Preprocessor}) as a comment. To
run the above script and assuming that it's executable
(chmod~+x~hello.eic):

\begin{quote}
\begin{verbatim}
% ./hello.eic
 ******* Hello from EiC's script mode. ******
%
\end{verbatim}
\end{quote}

Another example of an EiC script is given in \T{script1.eic}:

\begin{quote} 
\begin{verbatim}
 1     #!/usr/local/bin/eic -f 
 2     #include <stdio.h>
 3
 4     // example of control of flow
 5     int i;
 6     int isqr(int x) { return x*x; }
 7     for(i=0;i<4;i++)
 8             printf("%d^2 =  %d\n",i,isqr(i));
 9     switch(i) {
10             case 4: printf(" good\n\n"); break;
11             default: printf(" bad\n\n");
12     }
13     // example of some file stuff;
14     // read in some tools
15     #include "tools/nxtString.c"
16     FILE *fp = fopen(_Argv[0],"r");
17     char *p;
18     while((p=nxtString(fp)))
19             printf("%s ",p);
20     fclose(fp);
21     printf("\n\n");
22     // further example of using command line args
23     if(_Argc) { // this is always true
24             int k=0;
25             printf("Processing command line arguments\n");
26             for(k=0;k<_Argc;k++) {
27                    printf("%s\n",_Argv[k]);
28             }
29    } else
30             printf("OOPS, an internal error has occurred\n");
\end{verbatim}
\end{quote}

An EiC shell script is interpreted from the top to the bottom. First
the code is compiled to byetcode, in its entirety, and then run. After
this, control will be parsed to the \T{main} function if it
exists. However, it is not illegal to have a script that does not
include the definition of a \T{main} function. If the EiC directive
\T{:exit} (as discussed on pg:~\pageref{item:exit}) is present in such
a script it will cause it to halt at the position
\T{:exit} is encounted and nothing will happen other than having the code
upto \T{:exit} compiled and parsed but it will not have been
executed.  Generally, the code for a fuction is not executed until it
is called, see line 8. Command line arguments are passed into to the
global variables \T{\_Argc} and
\T{\_Argv}, see lines 16 and 23 to 30. For example:
\begin{quote} 
\begin{verbatim}
% script1.eic  abc 123 -DHELP
\end{verbatim}
\end{quote}
Implies that:
\begin{quote} 
\begin{verbatim}
_Argc = 4,                 _Argv[0] = "sript1.eic"    
_Argv[1] = "abc"           _Argv[2] = "123"
_Argv[3] = "-DHELP"        _Argv[4] = NULL
\end{verbatim}
\end{quote}



To alter the behaviour of EiC during script-mode just add the
appropriate switch to the first line of the script, as seen for \T{-f}
on line 1 above. 


\subsection{Embedding or linking to EiC}
\label{sec:embedding}
\index{EiC!embedding}
\index{embedding EiC}
\index{address operator}
\index{address specifier}

To Link against EiC you first need to build the source distribution.
Then linking to EiC from aother programs is done by linking against
the EiC libraries (\T{libeic} and \T{libstdClib}) in \T{EiC/lib}. In
the directory \T{EiC/main/examples} there is an example program called
\T{embedEiC.c} that links to EiC. Build and run it from the
\T{EiC/main/examples} directory by entering (assuming EiC has been
installed in \T{/usr/local/EiC}):

\begin{production}
\begin{verbatim}
% gcc embedEiC.c -L/usr/local/EiC/lib -leic -lstdClib -lm  
% a.out
\end{verbatim}
\end{production}	
 For  communicating commands to EiC from another program there are two
functions supplied:
\begin{production}
\begin{verbatim}
int EiC_run(int argc, char **argv);
\end{verbatim}
\end{production}
 and
\begin{production}
\begin{verbatim}
void EiC_parseString(char *command, ...);
\end{verbatim}
\end{production}

The \T{EiC\_run} function is used to run C source files. The
\T{EiC\_parseString} function is used to pass C or preprocessor commands
to EiC via a string, such as:
\begin{production}
\begin{verbatim}
EiC_parseString("#include <stdio.h>");
EiC_parseString("int a = 10,i;");
EiC_parseString("for(i=0;i<a;i++)"
                "	printf(\"%%d\\n\",i);");
\end{verbatim}
\end{production}
At present the main facility for sharing data between EiC and
other applications is via the address operator \T{@}:

\begin{quote}
\begin{verbatim}
int a @ dddd;
\end{verbatim}
\end{quote}

The above defines \T{a} to be an integer and is stored at address \T{dddd},
which must be an integral constant. The constant address \T{dddd} is
not simply an address conjured up. Its purpose is to enable access to
data, or even functions, defined in compiled code. 

When applied to function definitions, the limitation at this stage is
that the function must take void arguments and return void:

\begin{production}
\begin{verbatim}
     void foo(void) @ dddd; 
\end{verbatim}
\end{production}

The above defines \T{foo} to be a builtin function located at address
\T{dddd}. For example:

\begin{quote}
\begin{verbatim}
int foo[5] = {1,2,3,4,5};
void fooey(void) {printf("fooey called\n");}
     ....
EiC_parseString("int foo[5] @ %ld;", (long)foo);
EiC_parseString("void fooey(void) @ %ld;", (long)fooey);
\end{verbatim}
\end{quote}

Further, \verb+int foo[5] @ 1256;+ defines \T{foo} to be an array of 5
ints mapped at the specified virtual address and the usual pointer
safety rules apply; that is, \verb/foo[5];/ will be caught 
as an illegal operation.


Also, you can pass in data to EiC via setting variables and you can
get EiC to output data to a file. In a future release of EiC, more 
facilities are expected to be added for sharing data between EiC and its
embedding system.

With respect to \T{EiC\_run}, to run the file "myfile.c"  and pass it the
command line arguments "hello" and "world", the following sequence of
commands would be used.

\begin{production}
\begin{verbatim}
char *argv[] = {"myfile.c", "hello", "world"};
int  argc = sizeof(argv)/sizeof(char*);
EiC_run(argc, argv);
\end{verbatim}
\end{production}


\subsubsection{EiC internet programming}
\label{sec:internet}
\index{EiC!CGI scripts}
\index{internet programming}
\index{CGI!programming}

It also possible to run \T{CGI}, Common Gateway Interface, scripts via
EiC. The function \T{virtualhtml.eic.cgi} in directory
\T{EiC/module/cgihtml/cgi-bin} is used to demonstrate this
facility:

\begin{quote}
\begin{verbatim}
#!/usr/local/bin/eic -f
:-I/usr/local/EiC/include
#include <stdio.h>
printf("Content-type: text/html\n\n");
printf("<html>\n"
       "<head><title> Simple Virtual HTML </title> </head>\n"
       "<body>\n"
       "<h1>Virtual HTML</h1><hr>\n"
       "Hey look, I just created this page virtually!!!\n");
printf("<p>Date:  %s <br> Current Time: %s\n",__DATE__,__TIME__);
printf("</body></html>\n");
\end{verbatim}
\end{quote}
Move the function to your \T{cgi-bin}\index{CGI!cgi-bin} and
run it from your browser by entering the following
line in the browser location window and then press enter:
\begin{quote}
\begin{verbatim}
http://www.your_domain/your-cgi-bin/virtualhtml.eic.cgi
\end{verbatim}
\end{quote}
The output to the browser window, other than a different date and
time, should be:
\begin{quote}
\begin{verbatim}
Virtual HTML

Hey look, I just created this page virtually!!! 

Date: Apr 18 1998 
Current Time: 14:25:35 
\end{verbatim}
\end{quote} 

You would have noticed on the 2nd line of \T{virtualhtml.cgi} the
command\newline \T{:-I/usr/local/EiC/include} (see
page~\pageref{item:comm-switch} for details on EiC search paths).
This is informing EiC to add the directory \T{/usr/local/EiC/include}
to its search path and is used when looking for include files.  The
reason this must be explicitly stated in the script is because each
CGI script is run in its own shell, which is owned by
\T{httpd} or
\T{www} depending on how your webserver is setup and therefore, the
\T{HOMEofEiC} environmental variable will not have been set (see
\SecPage{RunningEiC}).  Note, this instruction is only needed if EiC is 
\T{not} installed in \T{/usr/local} or \T{/usr}.


\subsubsection{EiC debugging CGI scripts}
\label{sec:debuggingCGI}
\index{EiC!CGI debugging}
\index{CGI!debugging}

To help debug your C-CGI scripts, EiC provides the command line switch
\T{-e}\index{command line switch!-e}. It, among other things, tells EiC
to inform Netscape as early as possible that the incoming
\T{content-type} is \T{text/plain}. This turns the browser window 
into a simple text-like shell which enables output from EiC
to be viewed in the usual way:

\begin{quote}
\begin{verbatim}
#!/usr/local/bin/eic -fe 
#include <stdio.h>
printf("Hello, world wide web!\n");
\end{verbatim}
\end{quote}

Now, as before, just call the above program, called \T{www1.eic.cgi},
from Netscape via:

\begin{quote}
\begin{verbatim}
http://www.your_domain/your-cgi-bin/www1.eic.cgi
\end{verbatim}
\end{quote}

The output in your browser's window should be:
\begin{quote}
\begin{verbatim}
Hello, world wide web!
\end{verbatim}
\end{quote}

Note also, if there was a bug or a syntax error in the your
cgi-script then EiC diagnostic messages would have quickly
pin-pointed the lines of code causing the problem. All the debugging
facilities of EiC, such as \T{trace} (see page~\pageref{item:trace})
and array-bound checking can now be used to help debug your
cgi-scripts and get the C part of your script running correctly and
within the browser environment.


\subsubsection{Running EiC interactively, non-interactively}
\label{sec:ScriptMode2}

A further method for running EiC non-interactively that is useful for
capturing an interactive session and can be used for reporting errors
encounted during such a session is to redirect EiC's standard input to
come from a file. The command line switch \T{-A}\index{command line
switch!-A} is of use here as it instructs EiC that all input from
stdin should be treated non-interactively, such as don't bother
performing bracket matching etc.  As this mode simulates an
interactive session all commands must be contained on a single line
and an explicit \T{:exit} must be used to end the session. It differs
from EiC's script mode (as discussed above) because each line is
compiled into bytecode and run individually; that is, each line is
executed as it is encounted. To input commands that require multiple
lines, use either the backslash character at the end of each line to
continue or use file inclusion. An example of such a script is given
in
\T{examples/hello.lst}:

\begin{quote}
\begin{verbatim}
#include <stdio.h>
#define str(x)  #x
#define xstr(x) str(x)
#define W  world!
printf(str(Hello) ", " xstr(W) "\n");
:exit   // DO NOT FORGET TO EXIT
\end{verbatim}
\end{quote}

And is run via:
\begin{quote}
\begin{verbatim}
% eic -As < hello.lst
\end{verbatim}
\end{quote}

The \T{-s}\index{command line switch!-s} switch is used to suppress
EiC startup messages. The important point here is not to forget to
add the exit directive,
\T{:exit}, to finish execution. Otherwise, EiC will get caught in an
infinite loop and you will have to use control Z to pause it and
the shell to kill its process.

\section{The EiC interpreter}
\label{sec:EiCInterpreter}

In addition to C preprocessor directives (as explained in
\ChpPage{Preprocessor}) the user can enter three main types
of input as given by the following
grammar\footnote{Appendix~\ref{chp:Syntax} provides an explanation for
the notation used to explain the EiC language}:

\small
\begin{quote}
\bp{
parse:\\
\>\T{:}eic-command  parse\\  
\>ext-decl parse\\
\>stmt  parse \\
\>\T{DONE}
}
\end{quote}
\normalsize

Note, the phases of EiC's translations are discussed in
\SecPage{PhasesTranslation}.



\subsection{EiC commands}
\label{sec:EiCcommands}
An interpreter directive as opposed to a C statement, a preprocessor
directive, or an external declaration (see below) are communicated from 
lines beginning with a colon \T{:} followed by an {\it
eic-command} production such as:

\small
\begin{quote}
\bp{
eic-command: \\
\>\T{show}  id \\
\>\T{rm} item\U{[}\T{,}item\U{]$^*$}\\
\>\T{clear} file-name\U{[}\T{,}file-name\U{]$^*$} \\
\>\T{gen} header-file \U{[}num\U{]}  \U{["}outfile\U{"}\U{]} \\
\>\T{exit}  \\
\>\T{variables} \\
\>\T{variables}  \U{[} type-name $|$ \T{tags} \U{]} \\
\>\T{help} \\
\>\T{history} \\
\>\T{files} \U{[} file-name \U{]} \\
\>\T{reset} \U{[} \T{here} \U{]} \\  
\>\T{-} comm-switch \U{[} path \U{]} \\
\>toggle \\
comm-switch: \T{one of}\\
\>\T{I  R  L} \\
toggle: \\ 
\>\T{trace} \U{[}\T{funcs}\U{]}\\
\>\T{listcode}  \U{[}\T{linenums}\U{]}\\
\>\U{[} \T{timer} $|$ \T{showline}  $|$ \T{interpreter}  $|$ \T{memdump} $|$ \T{verbose} $|$ \T{includes} \U{]} \\
item: \T{one of}\\
\>identifier  constant-expression \\
path:\\
\>\T{any valid directory path} \\
file-name\\
\>\T{the name of any included file}
} 
\end{quote}
\normalsize


For example, you exit\index{exit EiC } EiC by entering {\tt
:exit}. Note, the keywords; that is, the terminal symbols used in the
{\it eic-command} productions will not conflict with C identifiers,
as the interpreter distinguishes the difference based upon context of
use (see \SecPage{NameSpace}). The {\it eic-command's} are now
explained:

\small
\begin{Ventry}{comm-switch }

\item[show] 
\label{item:show}
\index{EiC command!show}
is used to display type and other information concerning
variable and function definitions and declarations. It also
provides a quick way to test for the existence of a
variable. Example: 

\begin{quote}
\begin{verbatim}
EiC 1> float (*fval)[5];
        (void)
EiC 2> :show fval
fval -> * ARY[5]float 
\end{verbatim}
\end{quote}


which is read as: fval is a pointer to an array of 5 floats. When \T{show}
is used to display a \T{structure} or \T{union} it reveals the size
and the members also:

\begin{quote}
\begin{verbatim}
EiC 3> struct stag {int x; double da[5];} st;
        (void)
EiC 4> :show st
st -> struct: size 44 bytes
        x -> int 
        da -> ARY[5]double 
\end{verbatim}
\end{quote}

If the structure or union contains nested structures or unions,
\T{show} only expands the first level of nesting:

\begin{quote}
\begin{verbatim}
EiC 5> struct { float v; struct stag st;} sr;
        (void)
EiC 6> :show sr
sr -> struct: size 48 bytes
        v -> float 
        st -> struct: size 44 bytes
                x -> int 
                da -> ARY[5]double 
\end{verbatim}
\end{quote}
 

\T{Show} can also be used to look at function declarations
and definitions:

\begin{quote}
\begin{verbatim}
EiC 7> int sqr(int x);
        (void)
EiC 8> :show sqr
sqr -> dec_Func (
        x: int 
        ) returning int 
        (void)
\end{verbatim}
\end{quote}

This is interpreted as: \T{sqr} is a function declaration \T{dec\_Func} that
receives an integer argument declared with name {\tt x} and returns
an integer to its caller. The prefix \T{dec\_} implies that the body
of the function has not yet been defined. Converting the declaration
into a definition:

\begin{quote}
\begin{verbatim}
EiC 9> int sqr(int x) { return x * x;}
        (void)
EiC 10> :show sqr
sqr -> Func (
        x: int 
        ) returning int 
        (void)
\end{verbatim}
\end{quote}

In EiC there are basically two types of functions (see section
\SecPage{FunctionTypes}). There are interpreter
functions\index{function!interpreter} and there are builtin
functions\index{function!builtin}. To distinguish these forms the
\T{show} command adds the prefix
\T{Builtin} to builtin functions:

\begin{quote}
\begin{verbatim}
EiC 11> :show printf
printf -> Builtin Func (
        ) returning 
\end{verbatim}
\end{quote}

Notice that the argument list is empty and the returning type is
undefined. The above informs that while the function \T{printf}, which
is discussed on page \pageref{item:printf}, is built
into EiC, it has not yet been prototyped.  No builtin function can be
called from EiC until its prototype has been processed:

\begin{quote}
\begin{verbatim}
EiC 12> printf("hello\n");
Error in ::EiC:: near line 12: Incorrect function usage: printf
\end{verbatim}
\end{quote}

However, this is easily rectified by including the appropriate header
file:

\begin{quote}
\begin{verbatim}
EiC 13> #include <stdio.h>
EiC 14> :show printf
printf -> Builtin Func (
        fmt: * const char ,
        ...
        ) returning int 
\end{verbatim}
\end{quote}

Now, a call can be made to \T{printf}:

\begin{quote}
\begin{verbatim}
EiC 15> printf("hello\n");
hello
        6
\end{verbatim}
\end{quote}

The 6 is the return value from \T{printf}
and represents the number of characters printed.

The \T{show} command also helps to promote function
documentation:\index{documentation}\index{function!documentation} as it
displays the first comment \T{past} the line the opening {\tt \{}
bracket of the function is on. Therefore, it provides a simple way of
adding function usage. For example, consider the following function
stored in \T{examples/regline.c}:

\begin{quote}
\begin{verbatim}
void regline(float *x,float *y, int n,     
             float *m,  float *b )
{
    /*  In a least square sense, find the equation:
     * y = mx + b; Returns the slope in `m' and
     * the offset in `b', from the data given in
     * `y' and `x'. `n' being the size of the
     * arrays `y' and `x'.
     */
     ...
\end{verbatim}
\end{quote}

Now from EiC:

\begin{quote}
\begin{verbatim}
EiC 16> #include examples/regline.c
EiC 17> :show regline
regline -> Func (
        x: * float ,
        y: * float ,
        n: int ,
        m: * float ,
        b: * float 
        ) returning void 

    /*  In a least square sense, find the equation:
     * y = mx + b; Returns the slope in `m' and
     * the offset in `b', from the data given in
     * `y' and `x'. `n' being the size of the
     * arrays `y' and `x'.
     */
\end{verbatim}
\end{quote}

Because of semantic reasons the comment considered to be the
documenting comment will be the {\it first comment} after the line
the opening bracket of the function is on. If the first comment
happens to start on the same line as the opening bracket it will not
be recognised and the next comment (if it exists) will be used to form
the documenting comment. See also the EiC command
\T{listcode} on page \pageref{item:listcode} for further examples of
\T{show}.

\item[rm] 
\label{item:rm}
\index{EiC command!rm}
        is used to remove mostly variables and functions from EiC's
        symbol table. Example:

\begin{quote}
\begin{verbatim}
EiC 18> :rm fval
        (void)
EiC 19> :show fval
Error in ::EiC:: near line 19: Unknown identifier fval
\end{verbatim}
\end{quote}
        Here the error is simply informing
        us that the identifier {\tt fval} is no 
        longer recognized by the interpreter.

        The operand to the \T{rm} operator can also be an integral
        {\it constant-expression} (see~\SecPage{Constants}). The value
        of the {\it constant-expression} is used when removing
        manually memory leaks as reported by \T{memdump}. It is an
        error to attempt to remove a memory item that is not deemed to
        be a memory leak or to use an invalid item number (see
        \T{memdump} page:~\pageref{item:memdump}).


\item[clear] 
\label{item:clear}
\index{EiC command!clear}
        is used to remove the contents of entire files from EiC's
        symbol tables and memory pool. This handy operator removes the
        contents of an entire file before say re-including it so
        as to avoid conflicts between variable and function changes
        that EiC forbids. Example:

\begin{quote}
\begin{verbatim}
EiC 1> #include "examples/sqr.c"
        (void)
EiC 2> :clear examples/sqr.c
\end{verbatim}
\end{quote}
The {\it file-name}, operand, must match with one of the strings listed
by the \T{:files} operator (page:~\pageref{item:files}). The \T{:clear} operator also
excepts a comma seperated list of file names, with no white spaces intervening.

\item[gen]
\label{item:gen}
\index{EiC command!gen}


The \T{gen} command takes upto three arguments, \textit{header-file},
\textit{num} and \textit{outfile}:

\small
\begin{quote}
\T{:gen} \textit{header-file} \U{[}\textit{num}\U{]}  \U{["}\textit{outfile}\U{"}\U{]} 
\end{quote}
\normalsize


The \T{gen} command is used for generating EiC interfaces to builtin
C code. It purpose is to allow the easy interfacing of EiC to libraries
of C, and this is covered in more detail in \SecPage{interface_ccode}.

\begin{quote}
\begin{verbatim}
EiC > :gen foo.h
\end{verbatim}
\end{quote}


The above would generate the EiC interface to the \textit{header-file}t
\T{foo.h} to \T{stdout}. 

The \textit{outfile} is used for redirecting the output from \T{:gen} to a file rather
than \T{stdout}:

\begin{quote}
\begin{verbatim}
EiC > :gen foo.h "foo.c"
\end{verbatim}
\end{quote}

The above would be used to direct the output from \T{:gen} to the file
\T{foo.c}. Note the \textit{outfile} argument must be a string; that
is, enclosed in double quotes..

The \textit{num} option is a constant integer value, and is used to control 
the level of multiplexing callback code to generate. The default value
for \textit{num} is 1:

\begin{quote}
\begin{verbatim}
EiC > :gen foo.h 4 "foo.c"
\end{verbatim}
\end{quote}


\item[exit]
\label{item:exit}
\index{EiC command!exit}
         is used to terminate an EiC session.

\item[status] 
\label{item:status}
\index{EiC command!status}
        is used for inspecting the current status
of the EiC toggle switches (see below).

\item[variables] 
\label{item:variables}
\index{EiC command!variables}
        \T{variables} is like the \T{show} command
        (page~\pageref{item:show}), except rather than showing just
        one identifier it shows groups of identifiers. However, unlike the
        \T{show} command it does not expand structures and unions to
        reveal their members.

        Basically, there are three forms of the \T{variables}
        command. When entered on its own it displays all the declared
        identifiers. Generally, this will supply too much
        information. Therefore, to limit the information produced by
        \T{variables} it is possible to select various subsets by
        using one of the two other forms:

\begin{quote}
\bp {
\>      \T{variables}  \U{[} type-name $|$ \T{tags} \U{]}
}
\end{quote}

        \begin{Ventry2}{type-name }
        \item[type-name] The {\it type-name} specifier is discussed
        in detail in section~\SecPage{TypeNames}. But briefly
        it is used to display various identifier types such as, 
        all the pointers to integers:

\begin{quote}
\begin{verbatim}
EiC 20> int *p1, *p2, a, b,c;
EiC 21> :variables int *
p1 -> * int 
p2 -> * int 
\end{verbatim}
\end{quote}
        See the \T{show} command on page~\pageref{item:show} for
        an explanation of such output.
        
As an example of using \T{variables} to view all the functions of
a specified form, consider:

\begin{quote}
\begin{verbatim}
EiC 22> int f(char * s) { return 1;}
        (void)
EiC 23> int foo(const char * str);
        (void)
EiC 24> :variables  int (char *)
f -> Func (
        s: * char 
        ) returning int 
foo -> dec_Func (
        str: * const char 
        ) returning int 
system -> Builtin Func (
        str: * const char 
        ) returning int 
...
\end{verbatim}
\end{quote}
Note, {\it type-qualifiers} are not considered in the
matching processing and any matching builtin function will also
be displayed.


\item[tags] 
        The \T{tags} option is used to display the structure, union
        (\SecPage{structures}) and the
        enumeration (\SecPage{enumeration}) tags that have been
        declared:
\begin{quote}
\begin{verbatim}
EiC 25> struct stag {int x,y;};
EiC 26> enum etag {RED,GREE, BLUE};
EiC 27> :variables tags
etag -> enum 
stag -> struct: size 8 bytes
\end{verbatim}
\end{quote}
        \end{Ventry2}
        
\item[help]
\label{item:help}
\index{EiC command!help}
         is used to obtain a quick reference summary of the EiC
        interpreter commands:
\begin{quote}
\scriptsize
\begin{verbatim}
EiC 28> :help
-----------------------------------------------------------------------------
EiC-COMMAND          SUMMARY DESCRIPTION
-----------------------------------------------------------------------------
:-I path       Append path to the include-file search list.
:-L            List search paths.
:-R path       Remove path from the include-file search list.
:clear fname   Removes the contents of file fname from EiC.
:exit          Terminates an EiC session.
:files         Display the names of all included files.
:files fname   Summarize the contents of the included file `fname'.
:gen fname     Generates EiC interface of the included file `fname'.
:gen fname "outfile" Places the interface in outfile
:gen 4 fname   Generates EiC interface with 4 levels of multiplexing.
:help          Display summary of EiC commands.
:history       List the history of all input commands.
:includes      Display path of include files when loaded.
:interpreter   Execute input commands. By default it is on.
:listcode      List stack code.
:listcode linenums     List stack code with associated line numbers.
:memdump       Show potential memory leaks.
:rm  dddd      Remove memory item dddd, which is a constant integer value.
:rm   f        Removes f's definition from the symbol tables.
:show f        Shows type or  macro definition of `f'.
:showline      Show input line after macro expansion.
:status        Display the status of the toggle switches.
:timer         Time in seconds of execution.
:trace         Trace function calls and line numbers during code execution.
:trace funcs   Trace function calls only during code execution.
:variables     Display declared variables and interpreter-ed function names.
:variables tags       Display the tag identifiers.
:variables type-name  Display variables of type `type-name'.
:verbose       Suppresses EiC's copyright and warning messages on start up.
------------------------------------------------------------------------------
\end{verbatim}
\end{quote}
\end{Ventry}


\begin{Ventry}{comm-switch }
\item[files]
\label{item:files} \index{EiC command!files}

      The \T{files} command is used to get a list of the names
of all the include files currently entered into EiC:

\begin{quote}
\begin{verbatim}
EiC 29> :files
::EiC::
starteic.h
stdio.h
stdarg.h
math.h

...

fcntl.h
sys/fcntl.h
sys/types.h
../doc/regline.c
\end{verbatim}
\end{quote}

It is also possible to get a summary of the contents of any
particular  include file:

\begin{quote}
\begin{verbatim}
EiC 30> :files fcntl.h
FCNTLH_ -> #define FCNTLH_
open -> Builtin Func (
        path: * const char ,
        access: int ,
        ...
        ) returning int
creat -> Builtin Func (
        path: * const char ,
        amode: usigned short
        ) returning int
        (void)

\end{verbatim}
\end{quote}

The contents of an include file is summarised by first displaying
the declared macros followed by the global variables (if any), which are
in turn followed by the function definintions.

\item[reset]
\label{item:reset} \index{EiC command!reset}

	The \T{reset} operator is used to set EiC to a default
        internal state.  All allocated memory is freed, the contents
        of all include files and global variables, included and
        declared after the \T{reset} point, are removed. All previous
        global scalar variables defined prior to the \T{reset} point will have
        their values restored.

	The default \T{reset} point sets EiC to the point that is
        equivalent to starting EiC with the \T{-N} command line switch
        (see, \SecPage{starteich}):

\begin{quote}
\begin{verbatim}
EiC 1> :files
::EiC::
starteic.h
stdio.h
stdarg.h
sys/stdtypes.h
sys/stdio.h
stdlib.h
 ...           
        (void)
EiC 2> :reset
        (void)
EiC 3> :files
::EiC::
        (void)
\end{verbatim}
\end{quote}

	It is also possible to define the \T{reset} point by using the
	\T{here} operator:

\begin{quote}
\begin{verbatim}
EiC 1> int p = 66;
        (void)
EiC 2> :reset here
        (void)
EiC 3> p = 88;
        88
EiC 4> :reset    // set the reset point to the current state.
        (void)
EiC 5> p;
        66
\end{verbatim}
\end{quote}

	However, there is no guarantee that the reset state will be
	restored 100\%.  This is because, EiC, at this stage, will not
	retore the contents of arrays or structure/union members to
	their initial values. Also, if a pointer is pointing to some
	allocated memory that was allocated before the \T{:reset here}
	command was issued but freed before the \T{:reset} command,
	then the behaviour of that pointer will be undefined.

\item[comm-switch] 
\label{item:comm-switch} \index{EiC command!comm-switch}
        the {\it comm-switch\label{comm-switch}} 
        production is analogous
        to what is commonly known as a C program's command--line switch, which
        is an argument usually preceded by a dash \T{-}. Comm-switches are 
        used to modify
        the behaviour of EiC and its preprocessor. 
        The current valid switches are:

        \begin{Ventry}{xxx}
        \item[I]         \index{EiC command!comm-switch!I} 
        insert the given path into the preprocessors search list. 
                Used during file inclusion. Example:

\begin{quote}
\begin{verbatim}
EiC 31> : -I./tests
\end{verbatim}
\end{quote}


        Append the directory {\tt tests}, which is off the currently
        working directory to the search list.

        \item[R]          \index{EiC command!comm-switch!R} 
        
        remove the given path from the preprocessors search list.
                Example:

\begin{quote}
\begin{verbatim}
EiC 32> :-R ./tests
\end{verbatim}
\end{quote}


        \item[L]  \index{EiC command!comm-switch!L} 
 
        list the current search list. Example:

\begin{quote}
\begin{verbatim}
EiC 33> :-L
\end{verbatim}
\end{quote}

        The include search list is further discussed in section 
        \SecPage{Include}.

        \end{Ventry}

\item[history]
\label{item:history} \index{EiC command!history}

EiC automatically records each command line as entered from the user
in a history list. The default maximum length of the history list
\index{history list} is set at compile time and is normally 500
lines. Individual lines are of arbitrary length.  When the history
list is full old lines are removed from the top while the new command
line entries are entered from the bottom.

The user can go backwards through the history list by either pressing
the up arrow or by pressing control-p; or forward by pressing the
down arrow or control-n. Each line of history can be re-edited and
then re-entered by pressing the enter key, \verb+<CR>+. The entire current
history list is seen via:

\begin{quote}
\begin{verbatim}
EiC 34> :history
float (*fval)[5];
:show fval
struct stag {int x; double da[5];} st;
:show st
struct { float v; struct stag st;} sr;
:show sr
int sqr(int x);
...
:history
\end{verbatim}
\end{quote}

Note, the list has been truncated manually.
\end{Ventry}
\normalsize

EiC has several  keywords that associate with the {\it toggle} production,
\SecPage{EiCInterpreter}.  They are all toggle-switches \index{toggle-switch} 
that are either turned on or off.  That is, they are turned on by
entering their command once and turned off by reentering the same
command:
\begin{quote}
\begin{verbatim}
EiC 1> int i;
        (void)
EiC 2> :timer   // turn timer on
<time taken>: 0
EiC 3> for(i=0;i<100000;++i);
        (void)
<time taken>: 2.6
EiC 4> :timer  // turn timer off
\end{verbatim}
\end{quote}


The status of all the toggle-switches can be examined by using
the EiC command \T{status} (pg:~\pageref{item:status}). 
The toggle-switches  provide EiC with the following
optional features:

\small
\begin{Ventry} {interpreter} 

\item[trace]
\label{item:trace} \index{EiC command!option!trace}

The trace facility is a toggle-switch (see below) with an extra
production.  If on, \T{trace}, traces the function calls and line numbers
associated with a given translation unit and prints this information
to the screen. Consider the following nonsensical piece of code, which
is stored (without the line numbers) in the file
\T{examples/testtrace.c}:

\begin{quote}
\begin{verbatim}
 1      int f(void)
 2      {   int i;
 3          for(i=0;i<3;++i)
 4              if(i>2)
 5                  break;
 6          return i;
 7      }
 8      int g(void)
 9      {   int k = 0,i = 2;
10          while(i--)
11              k += f();
12          return k;
13      }
14      int main(void)
15      {   int i = 2;
16          do {
17              int k = g();
18          } while(--i);
19          return 0;
20      }
\end{verbatim}
\end{quote}

Now, \T{trace} can be used to follow the sequence of program flow:
\begin{quote}
\begin{verbatim}
EiC 1> #include examples/testtrace.c
EiC 2> :trace
EiC 3> main();
[main] 15,17,
[g] 9,10,11,
[f] 3,4,3,4,3,4,3,6,
[g] 11,10,11,
[f] 3,4,3,4,3,4,3,6,
[g] 11,10,12,
[main] 17,18,17,
[g] 9,10,11,
[f] 3,4,3,4,3,4,3,6,
[g] 11,10,11,
[f] 3,4,3,4,3,4,3,6,
[g] 11,10,12,
[main] 17,18,19,
[::EiC::] 
\end{verbatim}
\end{quote}

The first line of the response tells us that control started at
function \T{main} and then passed through lines 15 and 17 after
which, control was passed to function \T{g}. In function \T{g},
control passed through lines 9, 10 and 11 before entering function
\T{f}, and so on. After leaving function \T{f}, on line 6, control was
passed back to function \T{g} on line 11, etc. Finally, the trace
finished when control was returned back to the EiC command line. The
trace facility can also be used during batch mode operations via the command line switch \T{-t}\index{command line switch!-t} (see also
\SecPage{EiCCommandLineOptions}):
\begin{quote}
\begin{verbatim}
% eic -t examples/testtrace
...
\end{verbatim}
\end{quote}

Clearly, trace can help in debugging \index{debugging programs}
programs: it traces the activation and steps in a sequence of code. It
can be used to quickly locate sections of code that are causing
crashes or blockages. However, at times, this amount of information
can be too verbose, and therefore, the \T{trace} command has the
optional argument \T{funcs}:
\begin{quote}
\bp{
\T{trace} \U{[}\T{funcs}\U{]}
}
\end{quote}

When the extra argument is specified, the trace facility
prints out only the names of the functions entered:

\begin{quote}
\begin{verbatim}
EiC 4> :trace funcs
EiC 5> g(); 
[g] 
[f] 
[g] 
[f] 
[g] 
[::EiC::] 
EiC 6> :trace  // turn trace off
\end{verbatim}
\end{quote}

\item[listcode]
\label{item:listcode}  \index{EiC command!option!listcode}
\T{Listcode} will be essentially of interest to those people interested in the bytecode \index{bytecode}
produced by EiC. If \T{listcode} is toggled on then the bytecode for
the current command or translation unit (see
\SecPage{translationUnit}) will be displayed, non-recursively, to the
screen.  That is, it does not show the code for any associated
functions. By default, \T{listcode} is off. For example:

\begin{quote}
\begin{verbatim}
EiC 1> int i;
EiC 2> :listcode  // toggle on
   0:halt                
        (void)
EiC 3> i = 5;
   0:pushint  5          
   1:stoint   0 0        
   2:halt                
        5
EiC 4> :listcode  // toggle off
\end{verbatim}
\end{quote}

\T{Listcode} also affects the output produced from the EiC command
\T{show} (see pg:~\pageref{item:show}). For example,
 consider the \T{swap} function as stored (without the line numbers)
in the file examples/swap.c:

\begin{quote}
\begin{verbatim}
 1      void swap(int *a, int *b)
 2      {
 3          /* swap the values of a and b */
 4          int t = *a;
 5          *a = *b;
 6          *b = t;
 7      }
\end{verbatim}
\end{quote}

Now from within EiC:
\begin{quote}
\begin{verbatim}
EiC 5> #include examples/swap.c
EiC 6> :show swap
swap -> Func (
        a: * int ,
        b: * int 
        ) returning void 

    /*  swap the values of a and b */

EiC 7> :listcode
   0:halt                
        (void)
EiC 8> :show swap
swap -> Func (
        a: * int ,
        b: * int 
        ) returning void 

    /*  swap the values of a and b */

   0:checkar  1 0           7:drefint             
   1:rvalptr  -1 1          8:refint              
   2:drefint                9:rvalptr  -2 1       
   3:stoint   0 1          10:bump     1          
   4:rvalptr  -1 1         11:rvalint  0 1        
   5:bump     1            12:refint              
   6:rvalptr  -2 1         13:eicreturn           
\end{verbatim}
\end{quote}

\T{listcode}, like the \T{trace} command (page~\pageref{item:trace}) has
an extra form: \verb+`listcode linenums'+. When used in this form it
displays the associated line numbers corresponding to the
bytecode\index{bytecode} instruction:

\begin{quote}
\begin{verbatim}
EiC 9> :listcode linenums  // toggle linenums on 
   9:   0:halt                

        (void)
EiC 10> :show swap
swap -> Func (
        a: * int ,
        b: * int 
        ) returning void 

    /*  swap the values of a and b */

   0:   0:checkar  1 0           5:   7:drefint             
   4:   1:rvalptr  -1 1          5:   8:refint              
   4:   2:drefint                6:   9:rvalptr  -2 1       
   4:   3:stoint   0 1           6:  10:bump     1          
   5:   4:rvalptr  -1 1          6:  11:rvalint  0 1        
   5:   5:bump     1             6:  12:refint              
   5:   6:rvalptr  -2 1          7:  13:eicreturn           

  10:   0:halt                

        (void)
EiC 11> :listcode  // toggle listcode and linenums off
\end{verbatim}
\end{quote}

Note, any line number with the value zero represents extra ``house
keeping'' code added by the EiC interpreter.

  


\item[memdump]
\label{item:memdump} \index{EiC command!toggle!memdump}
EiC attempts to keep track of all memory dynamically allocated. 
If EiC cannot find the owner of a piece of dynamic memory the
address and how it was allocated will show up automatically if
\T{memdump} is switched on. For example, consider the following
useless piece of code in \T{examples/leak.c}:

\begin{quote}
\begin{verbatim}
 1      #include <stdlib.h>
 2      void leak(void)
 3      {
 4          char * s = malloc(10);
 5      }
\end{verbatim}
\end{quote}

The following session provides an example of the usage of memdump:

\small
\begin{quote}
\begin{verbatim}
EiC 9> #include examples/leak.c
EiC 10> :memdump   // toggle memdump on
       (void)
EiC 11> leak();
       (void)
item 3656 Create line 917 file stdlib.c nbytes 10
\end{verbatim}
\end{quote}
\normalsize


The above output informs that the memory item number 3656, which
maybe different number during your session of length 10 bytes cannot be
associate with an owner. That is, it is a potential leak. Also, it
tells us that the memory was allocated from line 917 in file
\T{stdlib.c} rather than from line 4 in file
\T{examples/leak.c}. Strickly speaking, it is correct because if we
were to look at line 917 in file \T{stdlib.c} we would find that this
is where EiC does its memory allocation for the EiC interpreter.

It is possible to remove leaked memory items via the
\T{rm} command (page:~\pageref{item:rm}):
\small
\begin{quote}
\begin{verbatim}
EiC 12> :rm 3656
       (void)
EiC 13> 
\end{verbatim}
\end{quote}
\normalsize

Also, note from the following session:
\small
\begin{quote}
\begin{verbatim}
EiC 1> :memdump  // toggle memdump on
        (void)
EiC 2> char *s = malloc(10);
        (void)
item 3611 Create line 781 file stdlib.c nbytes 10
EiC 3> free(s);
        (void)
\end{verbatim}
\end{quote}
\normalsize

shows that all memory allocated dynamically in EiC is considered to be a
potential leak -- if not freed. This is because EiC does not look at
the memory a pointer is pointing to when assigning homes to dynamic
memory items. Therefore, the usage of \T{memdump} should be considered
carefully when trying to determine if a  genuine memory leak has
occured or not. However, \T{memdump} is still useful as it  provides a
guide to locating potential memory leaks.

\item[timer]
\label{item:timer} \index{EiC command!toggle!timer} \index{timer}

if on, the execution time in seconds of a given translation unit is
printed out.  By default, the timer is off. From the following piece
of code I get from my 66Mhz 486 PC:
\begin{quote}
\begin{verbatim}
EiC 1> int i; float f,g;
EiC 2> :timer   // turn timer on
EiC 3> for(i=0;i<200000;++i) f *=g;
        (void)
<time taken>: 8.57
EiC 4> :timer  // turn timer off
\end{verbatim}
\end{quote} The \T{timer} is handy when attempting to optimise \index{optimise} 

a piece of code as it measures the actual processor time used.

\item[interpreter] \label{item:interpreter} \index{EiC command!toggle!interpreter}
         if on then input commands will be
         interpreter-ed. By default it is on.
\item[showline]
\label{item:showline} \index{EiC command!toggle!showline}

if on then the input sequence to the
interpreter is displayed. Useful for
inspecting the expansion of macros. 
By default it is off:

\begin{quote}
\begin{verbatim}
EiC 1> #define help(s)  system("man 3 " #s)
        (void)
EiC 2> :showline   // toggle showline on
        (void)
EiC 3> help(printf);       
system("man 3 " "printf");
...
EiC 4> :showline  // toggle showline off
\end{verbatim}
\end{quote}     
This facility can also be turned on from the command line using the
switch \T{-p}\index{command line switch!-p}:
\begin{quote}
\begin{verbatim}
 % eic -p
\end{verbatim}
\end{quote}     


\item[verbose]
\label{item:verbose}\index{EiC command!toggle!verbose}

the verbose command is essentially used when running EiC remotely, as
it suppresses EiC's copyright and warning messages on start up. It can
also be turned on from the command line using the switch \T{-v}\index{command line switch!-v}
\begin{quote}
\begin{verbatim}
 % eic -v
\end{verbatim}
\end{quote}     

\item[includes]
\label{item:includes} \index{EiC command!toggle!include}
If on, the path of include files will be displayed when loaded. This
facility can also be turned on from the command line using 
the \T{-P}\index{command line switch!-P} switch:
\begin{quote}
\begin{verbatim}
 % eic -P
\end{verbatim}
\end{quote}     
and causes EiC to reveal the paths of all the files it
includes.

\end{Ventry}
\normalsize














