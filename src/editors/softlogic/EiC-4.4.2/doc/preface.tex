\chapter{Preface}

EiC was developed from a perceived need for a complete interactive 
C interpreter -- or more correctly, an interactive bytecode C
compiler that can also run non-interactively in 
batch mode style. The main advantages of this latter mode is that
EiC can compile and run C programs without leaving around 
executables or object code and that there is no concern about
which platform the code is run on.

EiC is designed to be a production tool, it is not to be viewed 
as a toy and is certainly one of the most complete C interpreters
built to date. It is suitable as: an aid in teaching C, for fast
prototype of new programs and as a research tool --- as it allows the
user to quickly interface and experiment with user supplied, standard
ISO C and POSIX.1 functions, via immediate statements, which are
statements that are executed immediately.

However, like EiC, this documentation is also at the beta stage; for
example, its library section is still under development and the
section on how to \T{e}xtend EiC, by adding new builtin
functionality, has yet to be documented.  Therefore, any contributions
or suggestions are certainly welcome, and can be sent to me at
Ed.Breen@Altavista.net

\section*{Copyright}
\index{copyright}
 
\begin{enumerate}
\item
Permission is given to make a personal copy of this material is
given to anyone who is currently using EiC. Redistribution or the
making of multiple copies of this document must be done only with
explicit permission from the Copyright holder of EiC.

\item
This document and EiC is provided ``as is'' and without any express or
implied warranties, including, without limitation, the implied
warranties of merchantibility and fitness for a particular purpose

\end{enumerate}

\section*{Acknowledgements}

Thanks to Martin Gonda for his contribution to EiC's error recovery
module; Ross Leon Richardson's for permission to incorporate his
online quick reference guide to ``The C Standard Library''. Hugues
Talbot made early suggestions during EiC's development. EiC's type
specifier was modeled from lcc, which is available free of charge
\cite{fraser-hanson95}. Part of EiC's runtime library
support were derived from the Standard C library, (C), 1992 by
P.J. Plauger, published by Prentice-Hall and are used with permission.
Thanks to Eugene D. Brooks III for Beta testing EiC and for motivating
and supporting many new developments within EiC. In particular:
pointer qualifiers, pointer pragma directives and the address
specifier \T{@}. Thanks to Alf Clement for porting EiC to the HP
platform. Thanks to Jochen Pohl for porting EiC to NetBSD and for
contributing to EiC's makefile system. Jean-Bruno Richard developed
EiC's initial `:gen' command and paved the way for getting callbacks
to EiC from compiled code working.








